% Options for packages loaded elsewhere
\PassOptionsToPackage{unicode}{hyperref}
\PassOptionsToPackage{hyphens}{url}
\PassOptionsToPackage{dvipsnames,svgnames,x11names}{xcolor}
%
\documentclass[
  letterpaper,
  DIV=11,
  numbers=noendperiod]{scrartcl}

\usepackage{amsmath,amssymb}
\usepackage{iftex}
\ifPDFTeX
  \usepackage[T1]{fontenc}
  \usepackage[utf8]{inputenc}
  \usepackage{textcomp} % provide euro and other symbols
\else % if luatex or xetex
  \usepackage{unicode-math}
  \defaultfontfeatures{Scale=MatchLowercase}
  \defaultfontfeatures[\rmfamily]{Ligatures=TeX,Scale=1}
\fi
\usepackage{lmodern}
\ifPDFTeX\else  
    % xetex/luatex font selection
\fi
% Use upquote if available, for straight quotes in verbatim environments
\IfFileExists{upquote.sty}{\usepackage{upquote}}{}
\IfFileExists{microtype.sty}{% use microtype if available
  \usepackage[]{microtype}
  \UseMicrotypeSet[protrusion]{basicmath} % disable protrusion for tt fonts
}{}
\makeatletter
\@ifundefined{KOMAClassName}{% if non-KOMA class
  \IfFileExists{parskip.sty}{%
    \usepackage{parskip}
  }{% else
    \setlength{\parindent}{0pt}
    \setlength{\parskip}{6pt plus 2pt minus 1pt}}
}{% if KOMA class
  \KOMAoptions{parskip=half}}
\makeatother
\usepackage{xcolor}
\setlength{\emergencystretch}{3em} % prevent overfull lines
\setcounter{secnumdepth}{-\maxdimen} % remove section numbering
% Make \paragraph and \subparagraph free-standing
\makeatletter
\ifx\paragraph\undefined\else
  \let\oldparagraph\paragraph
  \renewcommand{\paragraph}{
    \@ifstar
      \xxxParagraphStar
      \xxxParagraphNoStar
  }
  \newcommand{\xxxParagraphStar}[1]{\oldparagraph*{#1}\mbox{}}
  \newcommand{\xxxParagraphNoStar}[1]{\oldparagraph{#1}\mbox{}}
\fi
\ifx\subparagraph\undefined\else
  \let\oldsubparagraph\subparagraph
  \renewcommand{\subparagraph}{
    \@ifstar
      \xxxSubParagraphStar
      \xxxSubParagraphNoStar
  }
  \newcommand{\xxxSubParagraphStar}[1]{\oldsubparagraph*{#1}\mbox{}}
  \newcommand{\xxxSubParagraphNoStar}[1]{\oldsubparagraph{#1}\mbox{}}
\fi
\makeatother


\providecommand{\tightlist}{%
  \setlength{\itemsep}{0pt}\setlength{\parskip}{0pt}}\usepackage{longtable,booktabs,array}
\usepackage{calc} % for calculating minipage widths
% Correct order of tables after \paragraph or \subparagraph
\usepackage{etoolbox}
\makeatletter
\patchcmd\longtable{\par}{\if@noskipsec\mbox{}\fi\par}{}{}
\makeatother
% Allow footnotes in longtable head/foot
\IfFileExists{footnotehyper.sty}{\usepackage{footnotehyper}}{\usepackage{footnote}}
\makesavenoteenv{longtable}
\usepackage{graphicx}
\makeatletter
\newsavebox\pandoc@box
\newcommand*\pandocbounded[1]{% scales image to fit in text height/width
  \sbox\pandoc@box{#1}%
  \Gscale@div\@tempa{\textheight}{\dimexpr\ht\pandoc@box+\dp\pandoc@box\relax}%
  \Gscale@div\@tempb{\linewidth}{\wd\pandoc@box}%
  \ifdim\@tempb\p@<\@tempa\p@\let\@tempa\@tempb\fi% select the smaller of both
  \ifdim\@tempa\p@<\p@\scalebox{\@tempa}{\usebox\pandoc@box}%
  \else\usebox{\pandoc@box}%
  \fi%
}
% Set default figure placement to htbp
\def\fps@figure{htbp}
\makeatother
% definitions for citeproc citations
\NewDocumentCommand\citeproctext{}{}
\NewDocumentCommand\citeproc{mm}{%
  \begingroup\def\citeproctext{#2}\cite{#1}\endgroup}
\makeatletter
 % allow citations to break across lines
 \let\@cite@ofmt\@firstofone
 % avoid brackets around text for \cite:
 \def\@biblabel#1{}
 \def\@cite#1#2{{#1\if@tempswa , #2\fi}}
\makeatother
\newlength{\cslhangindent}
\setlength{\cslhangindent}{1.5em}
\newlength{\csllabelwidth}
\setlength{\csllabelwidth}{3em}
\newenvironment{CSLReferences}[2] % #1 hanging-indent, #2 entry-spacing
 {\begin{list}{}{%
  \setlength{\itemindent}{0pt}
  \setlength{\leftmargin}{0pt}
  \setlength{\parsep}{0pt}
  % turn on hanging indent if param 1 is 1
  \ifodd #1
   \setlength{\leftmargin}{\cslhangindent}
   \setlength{\itemindent}{-1\cslhangindent}
  \fi
  % set entry spacing
  \setlength{\itemsep}{#2\baselineskip}}}
 {\end{list}}
\usepackage{calc}
\newcommand{\CSLBlock}[1]{\hfill\break\parbox[t]{\linewidth}{\strut\ignorespaces#1\strut}}
\newcommand{\CSLLeftMargin}[1]{\parbox[t]{\csllabelwidth}{\strut#1\strut}}
\newcommand{\CSLRightInline}[1]{\parbox[t]{\linewidth - \csllabelwidth}{\strut#1\strut}}
\newcommand{\CSLIndent}[1]{\hspace{\cslhangindent}#1}

\KOMAoption{captions}{tableheading}
\makeatletter
\@ifpackageloaded{caption}{}{\usepackage{caption}}
\AtBeginDocument{%
\ifdefined\contentsname
  \renewcommand*\contentsname{Table of contents}
\else
  \newcommand\contentsname{Table of contents}
\fi
\ifdefined\listfigurename
  \renewcommand*\listfigurename{List of Figures}
\else
  \newcommand\listfigurename{List of Figures}
\fi
\ifdefined\listtablename
  \renewcommand*\listtablename{List of Tables}
\else
  \newcommand\listtablename{List of Tables}
\fi
\ifdefined\figurename
  \renewcommand*\figurename{Figure}
\else
  \newcommand\figurename{Figure}
\fi
\ifdefined\tablename
  \renewcommand*\tablename{Table}
\else
  \newcommand\tablename{Table}
\fi
}
\@ifpackageloaded{float}{}{\usepackage{float}}
\floatstyle{ruled}
\@ifundefined{c@chapter}{\newfloat{codelisting}{h}{lop}}{\newfloat{codelisting}{h}{lop}[chapter]}
\floatname{codelisting}{Listing}
\newcommand*\listoflistings{\listof{codelisting}{List of Listings}}
\makeatother
\makeatletter
\makeatother
\makeatletter
\@ifpackageloaded{caption}{}{\usepackage{caption}}
\@ifpackageloaded{subcaption}{}{\usepackage{subcaption}}
\makeatother

\usepackage{bookmark}

\IfFileExists{xurl.sty}{\usepackage{xurl}}{} % add URL line breaks if available
\urlstyle{same} % disable monospaced font for URLs
\hypersetup{
  pdftitle={Best practices and practical applications for salmon data mobilization---a guide for salmon biologists assuming responsibility for data stewardship},
  colorlinks=true,
  linkcolor={blue},
  filecolor={Maroon},
  citecolor={Blue},
  urlcolor={Blue},
  pdfcreator={LaTeX via pandoc}}


\title{Best practices and practical applications for salmon data
mobilization---a guide for salmon biologists assuming responsibility for
data stewardship}
\author{Brett Johnson \and Scott Akenhead \and Jennifer Bayer \and Tomas
Bird \and Samuel Cimino \and Graeme Diack \and Lara Erikson \and Nancy
Leonard \and Catherine Michielsens \and Fiona Martens \and Emily
Lescak \and Gottfried Pestal \and Matt Jones \and Shirly
Stephen-Ambrose}
\date{}

\begin{document}
\maketitle


\textbf{Keywords:} Salmon data stewardship, Data interoperability, FAIR
principles, Persistent identifiers (PIDs), Controlled vocabularies,
Metadata standards, Application Programming Interface (API), Data
citation, Ontology development

\subsubsection{Abstract}\label{abstract}

Salmon research, management and conservation generates increasingly vast
and diverse data crucial for effective decision making for resource
management. Yet, these resources remain largely fragmented across
jurisdictions, disciplines and outdated infrastructures, limiting their
use in responsive fisheries management. Biologists are increasingly
assuming the responsibilities of data stewards to address these
challenges, yet often lack clear guidance or institutional support to do
so. We provide practical guidance for those transitioning into these
essential stewardship roles, outlining actionable best practices
specific to the salmon research and management domain, and argue that
effective salmon management hinges upon formally establishing data
stewardship as a dedicated, institutionally supported professional role.
We outline key best practices ---including both socio-cultural and
technical solutions that collectively ensure salmon data meet FAIR
(Findable, Accessible, Interoperable, Reusable) principles and respect
Indigenous Data Sovereignty. Through an illustrative case study
involving sockeye salmon productivity analyses across Pacific Coast
jurisdictions, we highlight how clearly defined stewardship practices
can enhance data reproducibility, integration, and management efficacy.
With a foundation of shared best practices, salmon data stewards will
enable faster, more transparent decisions that draw from broader,
cross-jurisdictional datasets, and support development of tools that
leverage recent advances in artificial intelligence---ultimately
strengthening the management and conservation of salmon populations.

\subsection{Introduction}\label{introduction}

Salmon biologists generate vast amounts of data on abundance, health,
and environmental conditions, yet these data remain fragmented,
inconsistently measured, and often incomplete across time, space, and
life history stages---limiting their value for robust research,
hypothesis evaluation, and management decision-making (Marmorek et al.
2011; Inman et al. 2021; Diack et al. 2024). Salmon traverse multiple
ecological regions and jurisdictional boundaries, resulting in data
collections managed by diverse agencies and institutions, often in
isolation. This fragmented data landscape undermines timely, integrated
analyses necessary for effective management and conservation decisions.
Additionally, the lack of institutional support and dedicated roles for
data management frequently relegates critical data stewardship tasks to
an ad hoc status---something performed off the side of a biologist's
desk. Institutional neglect of formal data stewardship has become a
bottleneck in adaptive salmon management and conservation efforts.

The growing complexity of fisheries management, combined with escalating
environmental uncertainties due to climate change, demands rapid,
integrated, and robust data analyses (Bull et al. 2022). Yet salmon
biologists transitioning into data stewardship roles typically receive
insufficient guidance or institutional support. We argue that fisheries
management agencies must formally acknowledge and fund dedicated data
stewardship roles to effectively mobilize and leverage salmon data.
Without this institutional commitment, data remain inaccessible and
fragmented, severely constraining the responsiveness and adaptability of
management actions. To address these critical shortcomings, we outline
practical steps that biologists, agencies and organizations can adopt.

\subsection{The Issue}\label{the-issue}

Effective integration and mobilization of salmon data mirrors the
complexity of salmon biology itself: these fish traverse freshwater,
estuarine, and marine ecosystems, crossing provincial, state, tribal,
federal, and international management boundaries (Groot and Margolis
1991). While localized successes in data coordination
exist---particularly within regional fisheries management offices and
treaty commissions---salmon data integrated across agencies for each
phase of the salmon life cycle is rare and prohibitively expensive for
all but the most pressing challenges. Most salmon datasets remain
confined within institutional silos, often undocumented, stored in
outdated systems, or formatted according to internal standards that are
incompatible with broader integration efforts. As a result, long-term
datasets critical to stock assessment and environmental monitoring
frequently become inaccessible, poorly understood, or effectively lost
once original data holders retire or move on.

This fragmentation is compounded by the number of disciplines and
organizations involved. Geneticists, oceanographers, freshwater
ecologists, stock assessment biologists, and fisheries managers all
contribute data using their own field-specific conventions and
workflows. Meanwhile, data is distributed across federal, state,
provincial, tribal, and academic institutions---each with its own
mandates, technologies, and metadata requirements. Many salmon
data-holding organizations rely on aging infrastructure or opaque,
undocumented standards that lag behind modern open-science practices.
This tangle of disciplinary and institutional fragmentation slows
integration, hinders reproducibility, and delays analyses that could
otherwise inform time-sensitive management decisions. Modernizing these
systems will require coordinated investment, grounded in shared
international data standards and stewardship practices that accommodate
the full disciplinary and geographic diversity of salmon science.

The consequences of inaction are already visible. When critical datasets
are hard to find, access, or interpret, biologists and analysts lose
valuable time trying to reconstruct or harmonize them. This reduces
transparency, increases the risk of errors, and delays urgent
conservation or management responses. Without clear accountability for
data stewardship, the system continues to rely on improvised,
inconsistent, and ultimately unsustainable practices.

\subsection{The Need for Coorindated
Action}\label{the-need-for-coorindated-action}

For fisheries managers, modernizing data systems and workflows is
essential to improve the quality, speed, and interoperability of
operational data assets. These systems must support an increasingly
complex decision-making landscape that now depends on integrating
broader types and sources of data, often in real time. At the same time,
researchers face pressure to generate insights on future salmon
abundance, the impacts of changing environmental conditions, and the
effectiveness of restoration strategies across all salmon life stages.
Yet the current scattered and siloed data landscape remains unfit for
purpose---both for science and for management.

Despite operating under different mandates, both researchers and
managers struggle to align their data with community-agreed principles
such as FAIR (Findable, Accessible, Interoperable, and Reusable)
(Wilkinson et al. 2016) and Indigenous Data Sovereignty frameworks like
the CARE principles (Collective Authority, Responsibility, and Ethics)
(Carroll, Rodriguez-Lonebear, and Martinez 2019; Jennings et al. 2023).
Adhering to CARE data management principles is all the more important
when it comes to salmon related data given the socio-cultural importance
of salmon to the Indigenous communities of the Northern Pacific and
Trans-Atlantic (Ween and Colombi 2013; Earth Economics 2021). Large
volumes of data collected through long-term monitoring programs hold
tremendous value, especially for secondary users---but are often
inaccessible due to a lack of time, resources, and incentives for data
producers to publish them (LINDENMAYER et al. 2012). Without clear
support and guidance, well-intentioned practitioners are left with ad
hoc approaches that limit reuse and interoperability. This gap can only
be bridged by equipping both data producers and stewards with tools,
support, and institutional backing to publish interoperable,
machine-readable metadata and datasets in alignment with shared
principles.

A coordinated approach to stewarding salmon data should follow
established open science standards and adhere explicitly to FAIR
principles, tailored specifically for salmon research and management
(Johnson and Stap 2024). Achieving meaningful interoperability demands
both breadth and depth. \textbf{Broad interoperability} integrates
diverse scientific domains, systems, and formats, requiring structured,
machine-readable data and metadata published openly for maximum
discoverability. \textbf{Deep interoperability} demands precise
definitions of salmon-specific terms and methods, ensuring data remains
meaningful and usable across contexts. Salmon data stewards can improve
conservation outcomes for salmon by coordinating across boundaries to
develop a shared foundation of data stewardship practices.

\subsection{Defining Data Stewardship in Salmon
Science}\label{defining-data-stewardship-in-salmon-science}

Data stewardship encompasses the coordinated practices, roles, and
responsibilities necessary to effectively manage, share, and reuse data
throughout its lifecycle. Within fisheries science, stewardship involves
ensuring data quality, compliance with agreed-upon standards, and the
establishment of clear governance to guide data collection,
documentation, integration, and preservation. However, salmon data
stewardship goes beyond mere technical data management; it involves
actively facilitating collaboration, communication, and
consensus-building among data producers and users across multiple
institutions and jurisdictions.

Specifically, effective salmon data stewards perform several critical
functions:

\begin{itemize}
\item
  \textbf{Technical oversight}: Ensuring metadata completeness,
  adherence to standardized terminologies and vocabularies, and robust
  quality assurance protocols.
\item
  \textbf{Social and organizational facilitation}: Leading stakeholder
  engagement, capacity-building activities, and negotiation of data
  access and sharing agreements, including addressing Indigenous
  Peoples' rights and interests in data governance.
\item
  \textbf{Institutional advocacy}: Championing the institutional
  recognition of data stewardship roles, promoting sustained investment
  and dedicated resources for data management infrastructure and
  practices.
\end{itemize}

A user-centered design approach to salmon data stewardship is critical
and focuses on creating tools that align with biologists' needs. Data
stewards play a critical role as business analysts, bridging the gap
between biologists and IT by translating data needs into application or
data system features. When data management is separated from biologists,
accountability weakens, and quality issues go unnoticed. While IT
expertise is essential for infrastructure and security, effective data
system design requires IT to act as an enabler, rather than gatekeeper,
provisioning self-serve data infrastructure. The Data Steward, serving
as a translator between IT and biologists, enables biologists to engage
independently with data systems, fostering ownership and accountability
and ultimately improving data quality for research and management.

Dedicated stewardship roles empower salmon biologists to bridge
disciplinary divides and jurisdictional barriers, transforming
fragmented datasets into cohesive, interoperable resources. By
proactively defining, implementing, and maintaining data standards and
workflows, salmon data stewards create conditions for timely, accurate,
and reproducible analyses. Such stewardship positions salmon biologists
to better inform adaptive management decisions, ultimately strengthening
salmon conservation and resilience.

\subsection{Updating Pacific-wide Sockeye Productivity: A Case Study for
What Agencies Could Do
Now}\label{updating-pacific-wide-sockeye-productivity-a-case-study-for-what-agencies-could-do-now}

This case study revisits a pacific coast wide sockeye productivity
dataset assembled from diverse agency sources by academic researchers
(Peterman and Dorner 2012). We reflect not on the significant work the
research team accomplished, but rather on the preventable institutional
and technical barriers that impeded their work---and continue to burden
data updates and reuse efforts today. Their study examined productivity
trends across 64 sockeye salmon stocks spanning Washington, British
Columbia (B.C.), and Alaska. However, attempting to replicate or build
upon this analysis today is an arduous, time-consuming, and error-prone
endeavour due to fragmented data sources, inconsistent formats, and lack
of standardized practices among the key institutions involved: the
Washington Department of Fish and Wildlife (WDFW), Fisheries and Oceans
Canada (DFO), and the Alaska Department of Fish and Game (ADF\&G).

Each section below highlights a key challenge faced by the team and
proposes practical steps based on our best practices
(Table~\ref{tbl-bestpractices}) that data-holding agencies could do to
enable easier integration, validation, and updating of salmon datasets
across jurisdictions and decades. This case study illustrates how
implementing the foundational concepts and practical recommendations
outlined in this paper can transform data stewardship practices within
these organizations. By doing so, they can significantly enhance data
accessibility, quality, and interoperability---ultimately enabling more
efficient and accurate analyses that support salmon conservation and
management.

\subsubsection{Challenge 1: Interpreting the Data --- What do these
numbers actually
mean?}\label{challenge-1-interpreting-the-data-what-do-these-numbers-actually-mean}

Peterman's team frequently worked with datasets that lacked basic
contextual information. Fields such as ``year,'' ``return,'' or ``age
class'' were often undefined or inconsistently used. For example, some
datasets recorded returns by calendar year while others used brood year,
and few included metadata to clarify the distinction. In many cases, the
team had to reconstruct metadata by back-checking against reports or
simulating assumptions (e.g., about age structure) to interpret the data
correctly.

\textbf{Remedies:}

\begin{itemize}
\item
  \textbf{Best Practice 4. Use Shared Data Models, Vocabularies and
  Metadata to Enable Integration.} To prevent this kind of ambiguity,
  agencies can now adopt internationally recognized metadata schemas
  such as ISO 19115 or Ecological Metadata Language, data models (Darwin
  Core Data Package) to model age and age type data concepts, and use
  controlled vocabularies to restrict the permissible values in the age
  field to calendar year, brood year, or otherwise.
\item
  \textbf{Best Practice 3: Make Data, People, Projects, and Outputs
  Discoverable, Linked and Citable with Persistent Identifiers (PIDs).}
  Assigning PIDs such as digital object identifiers (DOIs) to protocols,
  methods, and people (via ORCIDs) and linking them together using data
  stores and catalogues links data to its provenance and ensures that
  methods, context, and interpretation decisions are traceable.
\end{itemize}

\subsubsection{Challenge 2: Accessing and Using the Data --- Where is it
stored, and how do I get
it?}\label{challenge-2-accessing-and-using-the-data-where-is-it-stored-and-how-do-i-get-it}

The Peterman dataset was compiled from multiple files scattered across
email inboxes, regional offices, and gray literature. Data were stored
in inconsistent formats, lacked clear versioning, and were difficult to
discover outside of specific research networks. Even today, no API or
structured access mechanism exists to update or query the data
programmatically. As a result, researchers hoping to build on the
dataset may have to start from scratch.

\textbf{Remedies:}

\begin{itemize}
\item
  \textbf{Best Practice 5: Store Data in Ways That Others Can Easily
  Access and Use\\
  }Agencies can use open-access data repositories or their own
  institutional data repositories or catalogues that make data
  discoverable using PIDs and provide programmatic access to data
  possible using Application Programming Interfaces.
\item
  \textbf{Best Practice 2: Reuse Proven Infrastructure to Save Time and
  Increase Interoperability\\
  }Rather than developing bespoke data catalogues or repositories,
  agencies should adopt existing catalogues used beyond their own
  institution such as the Ocean Biodiversity Information System, Zenodo,
  or the Knowledge Network for Biocomplexity). These are proven
  platforms with a broad user base that support persistent storage,
  discoverability, and interoperability.
\end{itemize}

\subsubsection{Challenge 3: Sustaining the Dataset --- Who is
responsible, and why should I
contribute?}\label{challenge-3-sustaining-the-dataset-who-is-responsible-and-why-should-i-contribute}

Once Peterman and his team completed their analysis, no formal plan
existed for sustaining or updating the dataset. Responsibility for
ongoing maintenance fell informally to former students and
collaborators. Despite its national and international relevance, the
dataset was never adopted by an agency as a living product. Moreover,
the original data contributors often lacked incentives, support, or
recognition for their efforts---conditions that persist in many data
environments today.

\textbf{Remedies:}

\begin{itemize}
\item
  \textbf{Best Practice 1: Make Data Governance Explicit to Support
  Trust and Reuse\\
  }Agencies should define roles, responsibilities, and decision-making
  processes through formal governance mechanisms such as data product
  charters.\\
  \emph{Practical application:} Use a DACI or RACI framework to assign
  maintenance responsibility and ensure continuity across staff turnover
  and institutional change.
\item
  \textbf{Best Practice 6: Incentivize and Track Data Sharing and
  Reuse\\
  }Visibility, credit, and metrics are critical for motivating data
  sharing. Agencies can embed citation guidance in metadata and track
  dataset reuse through COUNTER-compliant dashboards or DataCite APIs.
\item
  \textbf{Best Practice 7: Build Community Through Co-Development and
  Mutual Benefit\\
  } Effective data stewardship requires collaboration between
  biologists, Indigenous communities, managers, and data professionals.
  Participatory design ensures that systems and standards meet user
  needs and are adopted over time.\\
  \emph{Practical application:} Facilitate cross-jurisdictional working
  groups to co-develop data standards and align on shared outcomes for
  priority datasets.
\end{itemize}

While the analytical contribution of the Peterman productivity dataset
remains significant, the barriers encountered in compiling,
interpreting, and maintaining the data are instructive. These challenges
are not unique to Peterman's team---they reflect systemic gaps in data
governance, documentation, infrastructure, and incentives. By adopting
the seven best practices outlined above, agencies and researchers can
transform legacy datasets into living resources, enabling
reproducibility, easing collaboration, and accelerating insight across
the salmon research and management community.

\begin{longtable}[]{@{}
  >{\raggedright\arraybackslash}p{(\linewidth - 4\tabcolsep) * \real{0.2123}}
  >{\raggedright\arraybackslash}p{(\linewidth - 4\tabcolsep) * \real{0.4788}}
  >{\raggedright\arraybackslash}p{(\linewidth - 4\tabcolsep) * \real{0.3066}}@{}}
\caption{Best practices and practical applications of salmon data
stewardship}\label{tbl-bestpractices}\tabularnewline
\toprule\noalign{}
\begin{minipage}[b]{\linewidth}\raggedright
\textbf{Best Practice}
\end{minipage} & \begin{minipage}[b]{\linewidth}\raggedright
\textbf{Summary}
\end{minipage} & \begin{minipage}[b]{\linewidth}\raggedright
\textbf{Practical Applications}
\end{minipage} \\
\midrule\noalign{}
\endfirsthead
\toprule\noalign{}
\begin{minipage}[b]{\linewidth}\raggedright
\textbf{Best Practice}
\end{minipage} & \begin{minipage}[b]{\linewidth}\raggedright
\textbf{Summary}
\end{minipage} & \begin{minipage}[b]{\linewidth}\raggedright
\textbf{Practical Applications}
\end{minipage} \\
\midrule\noalign{}
\endhead
\bottomrule\noalign{}
\endlastfoot
1. Make Data Governance Explicit to Support Trust and Reuse &
Establishing clear governance structures ensures quality,
accountability, and compliance with FAIR and CARE principles. It enables
trust and long-term stewardship across multi-organizational projects. &
\begin{minipage}[t]{\linewidth}\raggedright
- Document roles and responsibilities using a Data Product Governance
Charter and structured frameworks (e.g., DACI or RACI).\\
- Integrate CARE principles to respect Indigenous data rights.\\
- Form a governance or oversight committee to review data structures,
timelines, and agreements.\strut
\end{minipage} \\
2. Reuse Proven Infrastructure to Save Time and Increase
Interoperability & Leveraging existing platforms and technologies
reduces costs and improves long-term interoperability and
sustainability. & \begin{minipage}[t]{\linewidth}\raggedright
- Use domain-specific repositories like OBIS or GBIF.\\
- Publish and archive data with KNB or Zenodo.\strut
\end{minipage} \\
3. Make Data, People, Projects, and Outputs Discoverable, Linked and
Citable with PIDs & Persistent identifiers (PIDs) connect data with
researchers, institutions, and outputs---supporting data citation,
reuse, and automated attribution. &
\begin{minipage}[t]{\linewidth}\raggedright
- Encourage use of ORCID iDs for researchers.\\
- Use ROR IDs for institutions.\\
- Assign DOIs via DataCite for data packages.\\
- Embed DOIs in dashboards and metadata.\strut
\end{minipage} \\
4. Use Shared Data Models, Vocabularies and Metadata to Enable
Integration & Common vocabularies, metadata standards, and ontologies
support integration across systems and preserve semantic meaning. &
\begin{minipage}[t]{\linewidth}\raggedright
- Adopt ISO 19115, EML, or DataCite metadata standards.\\
- Model datasets using the Darwin Core Data Package.\\
- Use controlled vocabularies or ontologies with PIDs.\strut
\end{minipage} \\
5. Store Data in Ways That Others Can Easily Access and Use & Structured
and accessible data formats ensure usability, reduce wrangling, and
support integration with analytical tools and applications. &
\begin{minipage}[t]{\linewidth}\raggedright
- Provide APIs using FastAPI, Flask, or Django REST.\\
- Archive in trusted repositories (e.g., GBIF, FRDR, USGS).\\
- Use GitHub-Zenodo for DOI assignment and preservation.\strut
\end{minipage} \\
6. Incentivize and Track Data Sharing and Reuse & Recognizing data
contributors and tracking reuse promotes a culture of sharing and
supports professional recognition. &
\begin{minipage}[t]{\linewidth}\raggedright
- License data with CC-BY 4.0.\\
- Include citation text and visible credit fields.\\
- Use COUNTER metrics and DataCite APIs to monitor reuse.\\
- Encourage dataset citation in references.\strut
\end{minipage} \\
7. Build Community Through Co-Development and Mutual Benefit & Engaging
users early ensures tools and standards meet real-world needs and
enhances long-term stewardship. &
\begin{minipage}[t]{\linewidth}\raggedright
- Participate in RDA Salmon Interest Group.\\
- Facilitate workshops for metadata and vocabulary alignment.\\
- Support community-engaged research with tangible benefits.\strut
\end{minipage} \\
\end{longtable}

\subsection{Conclusion}\label{conclusion}

We envision a future where salmon data are not locked in filing
cabinets, buried in PDFs, or hidden in siloed spreadsheets, but instead
flow seamlessly through a well-stewarded digital ecosystem. In this
future, field biologists, Indigenous guardians, modelers, and
policymakers interact with a living knowledge system---one that is
transparent, discoverable, interoperable, and constantly evolving.
Metadata standards, controlled vocabularies, and shared governance
frameworks are not afterthoughts but integral to the culture of data
collection and use. Scientists receive credit for publishing
high-quality data, and users trust the provenance and structure of the
datasets they rely on to make critical management decisions.

The process of aggregating and understanding salmon data has been
fraught with challenges, from the complex life histories of the species,
changing habitat conditions, and human effects to the jurisdictional and
institutional silos that obstruct coordination. Salmon traverse
political and ecosystem boundaries, yet our data systems often do not.
The need of the hour is a strategy that is simple, flexible, and
coordinated---one that embraces the FAIR principles while integrating
the CARE principles to honor the sovereignty and knowledge systems of
Indigenous communities.

In the face of accelerating environmental change and rapid technological
innovation, a transformative opportunity exists: to harness fragmented
and heterogeneous datasets and translate them into standardized,
interoperable data products. These products---rooted in
community-endorsed standards and supported by persistent
infrastructure---will allow researchers and resource managers to adopt
data-intensive analytical approaches, automate updates to population
indicators, and integrate diverse datasets into real-time
decision-support tools. This marks a critical step toward transitioning
salmon science into the ``Fourth Paradigm'' of discovery(Hey 2012): one
driven by data-intensive, cross-disciplinary synthesis.

\subsubsection{Competing interests}\label{competing-interests}

\subsubsection{Acknowledgements}\label{acknowledgements}

\subsubsection{References}\label{references}

\phantomsection\label{refs}
\begin{CSLReferences}{1}{0}
\bibitem[\citeproctext]{ref-bull2022}
Bull, C D, S D Gregory, E Rivot, T F Sheehan, D Ensing, G Woodward, and
W Crozier. 2022. {``The Likely Suspects Framework: The Need for a Life
Cycle Approach for Managing Atlantic Salmon ({\emph{Salmo Salar}})
Stocks Across Multiple Scales.''} Edited by Wesley Flannery. \emph{ICES
Journal of Marine Science} 79 (5): 1445--56.
\url{https://doi.org/10.1093/icesjms/fsac099}.

\bibitem[\citeproctext]{ref-carroll2019}
Carroll, Stephanie Russo, Desi Rodriguez-Lonebear, and Andrew Martinez.
2019. {``Indigenous Data Governance: Strategies from United States
Native Nations.''} \emph{Data Science Journal} 18 (1): 31.
\url{https://doi.org/10.5334/dsj-2019-031}.

\bibitem[\citeproctext]{ref-diack2024}
Diack, Graeme, Tom Bird, Scott Akenhead, Jennifer Bayer, Deirdre Brophy,
Colin Bull, Elvira de Eyto, et al. 2024. {``Salmon Data Mobilization.''}
\emph{North Pacific Anadromous Fish Commission Bulletin}, December.
\url{https://doi.org/10.23849/npafcb7/x3rlpo23a}.

\bibitem[\citeproctext]{ref-eartheconomics2021}
Earth Economics. 2021. {``The Sociocultural Significance of Pacific
Salmon to Tribes and First Nations.''} Tacoma, Washington.
\url{https://www.psc.org/wp-content/uploads/wpfd/preview_files/The-Sociocultural-Significance-of-Salmon-to-Tribes-and-First-Nations(5da9942da9fb4fe0d77eb32bd6165e43).pdf}.

\bibitem[\citeproctext]{ref-groot1991}
Groot, Cornelis, and L. Margolis. 1991. \emph{Pacific Salmon Life
Histories}. UBC Press.
\url{https://books.google.ca/books?id=I_S0xCME0CYC}.

\bibitem[\citeproctext]{ref-hey2012}
Hey, Tony. 2012. {``The Fourth Paradigm {\textendash} Data-Intensive
Scientific Discovery.''} In, 1--1. Springer Berlin Heidelberg.
\url{https://doi.org/10.1007/978-3-642-33299-9_1}.

\bibitem[\citeproctext]{ref-inman2021}
Inman, Sarah, Janessa Esquible, Michael Jones, William Bechtol, and
Brendan Connors. 2021. {``Opportunities and Impediments for Use of Local
Data in the Management of Salmon Fisheries.''} \emph{Ecology and
Society} 26 (2). \url{https://doi.org/10.5751/ES-12117-260226}.

\bibitem[\citeproctext]{ref-jennings2023a}
Jennings, Lydia, Talia Anderson, Andrew Martinez, Rogena Sterling,
Dominique David Chavez, Ibrahim Garba, Maui Hudson, Nanibaa' A.
Garrison, and Stephanie Russo Carroll. 2023. {``Applying the {`}CARE
Principles for Indigenous Data Governance{'} to Ecology and Biodiversity
Research.''} \emph{Nature Ecology \& Evolution} 7 (10): 1547--51.
\url{https://doi.org/10.1038/s41559-023-02161-2}.

\bibitem[\citeproctext]{ref-johnson2024}
Johnson, Brett, and Tim van der Stap. 2024. {``Data Mobilization Through
the International Year of the Salmon Ocean Observing System.''}
\emph{North Pacific Anadromous Fish Commission Bulletin}, December.
\url{https://doi.org/10.23849/npafcb7/6a4ddpde4}.

\bibitem[\citeproctext]{ref-lindenmayer2012}
LINDENMAYER, DAVID B., GENE E. LIKENS, ALAN ANDERSEN, DAVID BOWMAN, C.
MICHAEL BULL, EMMA BURNS, CHRIS R. DICKMAN, et al. 2012. {``Value of
Long{-}Term Ecological Studies.''} \emph{Austral Ecology} 37 (7):
745--57. \url{https://doi.org/10.1111/j.1442-9993.2011.02351.x}.

\bibitem[\citeproctext]{ref-marmorek2011}
Marmorek, David, Darcy Pickard, Alexander Hall, Katherine Bryan, Liz
Martell, Clint Alexander, Katherine Wieckowski, Lorne Greig, and Carl
Schwarz. 2011. {``Cohen Commision Technical Report 6-Fraser River
Sockeye Salmon: Data Synthesis and Cumulative Impacts.''} Vancouver,
B.C. \url{http://www.cohencommission.ca/}.

\bibitem[\citeproctext]{ref-peterman2012}
Peterman, Randall~M., and Brigitte Dorner. 2012. {``A Widespread
Decrease in Productivity of Sockeye Salmon ({\emph{Oncorhynchus Nerka}})
Populations in Western North America.''} Edited by Jordan S. Rosenfeld.
\emph{Canadian Journal of Fisheries and Aquatic Sciences} 69 (8):
1255--60. \url{https://doi.org/10.1139/f2012-063}.

\bibitem[\citeproctext]{ref-ween2013}
Ween, Gro, and Benedict Colombi. 2013. {``Two Rivers: The Politics of
Wild Salmon, Indigenous Rights and Natural Resource Management.''}
\emph{Sustainability} 5 (2): 478--95.
\url{https://doi.org/10.3390/su5020478}.

\bibitem[\citeproctext]{ref-wilkinson2016}
Wilkinson, Mark D., Michel Dumontier, IJsbrand Jan Aalbersberg,
Gabrielle Appleton, Myles Axton, Arie Baak, Niklas Blomberg, et al.
2016. {``The FAIR Guiding Principles for Scientific Data Management and
Stewardship.''} \emph{Scientific Data} 3 (1): 160018.
\url{https://doi.org/10.1038/sdata.2016.18}.

\end{CSLReferences}

\subsection{Appendix 1. Real-world Example Applications of the Best
Practices}\label{appendix-1.-real-world-example-applications-of-the-best-practices}

Here we provide detailed descriptions of the seven best practices for
salmon data stewardship, along with practical applications and
real-world examples. This is not an exhaustive list, but rather a
starting point for salmon biologists and data stewards to implement
effective data stewardship practices in their work based on examples
from the salmon research and management community.

\paragraph{\texorpdfstring{\textbf{1. Make Data Governance Explicit to
Support Trust and
Reuse}}{1. Make Data Governance Explicit to Support Trust and Reuse}}\label{make-data-governance-explicit-to-support-trust-and-reuse}

Clear governance defines roles, responsibilities, and procedures
ensuring data quality, long-term maintenance, accountability, and
compliance with community principles such as FAIR and CARE. Effective
governance fosters trust, facilitates data sharing, and reduces
ambiguity regarding decision making, and is critical for coordinating
both technical and sociocultural aspects of data stewardship.

In collaborative international or multi-organizational settings,
establishing governance at the outset of a project is crucial for
aligning diverse groups, including biologists, data managers, Indigenous
communities, policymakers, and other participants. Early governance
planning should establish clear, collaborative frameworks that respect
each group's expertise and needs from the beginning.~

\subparagraph{Practical Applications:}\label{practical-applications}

Document roles and responsibilities clearly at project start using a
Project or Data Product Governance Charter and structured frameworks
(e.g., DACI or RACI charts) that relate to organizational data policies.

\begin{itemize}
\item
  \href{https://raw.githubusercontent.com/Br-Johnson/sdm-paper/refs/heads/main/examples/Cal_DMP.pdf}{Example
  of a Data Management Plan from the California Department of Water
  Resources}
\item
  \href{https://dmptool.org/public_templates?page=ALL}{Data Management
  Plan Templates}
\end{itemize}

Integrate CARE principles to ensure ethical governance and respect
Indigenous data rights.

\begin{itemize}
\tightlist
\item
\end{itemize}

Create a governance or oversight committee for regular data practice
reviews and decision making regarding data structures, timelines, data
sharing agreements and interoperability protocols

\paragraph{\texorpdfstring{\textbf{2. Reuse Proven Infrastructure to
Save Time and Increase
Interoperability}}{2. Reuse Proven Infrastructure to Save Time and Increase Interoperability}}\label{reuse-proven-infrastructure-to-save-time-and-increase-interoperability}

Building custom solutions should be avoided where possible. Maximizing
existing platforms and technologies reduces costs, accelerates
implementation, and increases data interoperability. Building modular,
interoperable systems grounded in proven technologies ensures
sustainable long-term stewardship.

\subparagraph{Practical Applications:}\label{practical-applications-1}

\begin{itemize}
\item
  \href{https://doi.org/10.14286/duc6mu}{Use the Ocean Biodiversity
  Information System or the Global Biodiversity Information Facility to
  standardize and host your data}
\item
  Use free data catalogue services such as the Knowledge Network for
  Biocomplexity (KNB) or Zenodo
\end{itemize}

\paragraph{\texorpdfstring{\textbf{3. Make Data, People, Projects, and
Outputs Discoverable, Linked and Citable with
PIDs}}{3. Make Data, People, Projects, and Outputs Discoverable, Linked and Citable with PIDs}}\label{make-data-people-projects-and-outputs-discoverable-linked-and-citable-with-pids}

Persistent identifiers (PIDs), including Digital Object Identifiers
(DOI) are essential for tracking the provenance and reuse of data, and
linking data, protocols, organizations and people. They allow for
consistent referencing, integration across systems, and automated credit
via data citation.

\subparagraph{Practical Applications:}\label{practical-applications-2}

\begin{itemize}
\item
  Encourage researchers to register for an
  \href{https://orcid.org/}{Open Researcher and Contributor ID (ORCID)}
  and include ORCIDs in metadata records and submission forms
\item
  Register your organization with the Research Organization Registry
  (ROR) and use ROR IDs to identify institutions involved in salmon
  science.
\item
  Assign DOIs to data packages, protocols, and reports using DataCite.
\item
  Embed DOIs in dashboards, figures, and metadata so they persist in
  derivative products.
\end{itemize}

\paragraph{\texorpdfstring{\textbf{4. Use Shared Data Models,
Vocabularies and Metadata to Enable
Integration}}{4. Use Shared Data Models, Vocabularies and Metadata to Enable Integration}}\label{use-shared-data-models-vocabularies-and-metadata-to-enable-integration}

Standardizing metadata and terminology ensures data can be interpreted
correctly and integrated across systems. Controlled vocabularies,
community ontologies, and structured metadata schemas allow data to
retain its full semantic meaning.

\subparagraph{Practical Applications:}\label{practical-applications-3}

\begin{itemize}
\item
  Configure data catalogues and metadata intake tools to accept
  Internationally recognized metadata schemas such as ISO 19115,
  Ecological Metadata Language (EML), or DataCite
\item
  Model datasets and databases using the Darwin Core Data Package model
\item
  Re-use or publish data terms that are shared online using a persistent
  identifier in a controlled vocabulary or ontology
\end{itemize}

\paragraph{\texorpdfstring{\textbf{5. Store Data in Ways That Others Can
Easily Access and
Use}}{5. Store Data in Ways That Others Can Easily Access and Use}}\label{store-data-in-ways-that-others-can-easily-access-and-use}

Making data easily accessible promotes its use in research and
management, enabling seamless integration with tools and applications.
Ensuring accessible, persistent data storage requires more than just
file hosting. Data should be structured, accessible via API, and stored
in repositories that support long-term preservation.

\subparagraph{Practical Applications:}\label{practical-applications-4}

\begin{itemize}
\item
  Provide Direct Data Access via Application Programming Interfaces
  (APIs) using tools such as FastAPI, Flask, or Django REST Framework
  that allows users to access, filter, and retrieve data
  programmatically, facilitating automation and integration into
  analytical tools and decision-support systems
\item
  Archive data in certified long-term, domain-specific repositories such
  as the Global Biodiversity Information Facility, the Federated
  Research Data Repository (FRDR), or NOAA's NCEI, USGS ScienceBase, or
  EMODnet
\item
  Leverage the integration between GitHub and Zenodo to automate
  archiving and DOI assignment, ensuring long-term data preservation.
\end{itemize}

\textbf{6. Incentivize and Track Data Sharing and Reuse}

The currency of research lies in recognition---credit, citations, and
opportunities for collaboration or co-authorship. Promoting data sharing
requires both cultural and technical infrastructure. By recognizing
contributions, tracking reuse, and supporting citation, data stewards
can create a system where sharing is rewarded.

\subparagraph{Practical Applications:}\label{practical-applications-5}

\begin{itemize}
\item
  License data for reuse using
  \href{https://creativecommons.org/licenses/by/4.0/deed.en}{Creative
  Commons Attribution 4.0 International}
\item
  Provide recommended citation text and visible credit fields in
  metadata
\item
  Create summary dashboards that highlight reuse using
  \href{https://www.countermetrics.org/}{COUNTER Code of Practice
  compliant metrics} to track dataset views/downloads and the DataCite
  APIs
\item
  Ensure that datasets are properly cited in journal articles using in
  text citations and the recommended citation in the articles list of
  references, not just in a Data Availability statement
\item
  Promote the view that well documented data publications are primary
  research outputs and are significant contributions to the field
\end{itemize}

\paragraph{\texorpdfstring{\textbf{7. Build Community Through
Co-Development and Mutual
Benefit}}{7. Build Community Through Co-Development and Mutual Benefit}}\label{build-community-through-co-development-and-mutual-benefit}

Creating an infrastructure that standardizes and provides cross-border
and cross-ecosystem data integration is only effective if there's
community engagement. Standards and tools must be co-developed with
their intended users using user-centred design principles (citation
required) to be effective. Engaging biologists, Indigenous stewards, and
data managers ensures relevance, usability, and long-term participation.

\subparagraph{Practical Applications:}\label{practical-applications-6}

\begin{itemize}
\item
  Participate in salmon data focussed communities such as
  the\href{https://www.rd-alliance.org/groups/salmon-research-and-monitoring-ig/}{Research
  Data Alliance's Salmon Research and Monitoring Interest Group}~
\item
  Run participatory workshops for metadata mapping and vocabulary
  alignment
\item
  Support and follow through on Community Engaged Research (citation
  required) that provides tangible value to the communities in which
  research or monitoring was conducted
\end{itemize}




\end{document}
